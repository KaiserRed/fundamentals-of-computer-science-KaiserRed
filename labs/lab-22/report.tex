\documentclass{article}

\usepackage[T2A]{fontenc}
\usepackage[english, russian]{babel}
\usepackage{graphicx} 
\graphicspath{ {./images/} }
\usepackage[letterpaper,top=2cm,bottom=2cm,left=3cm,right=3cm,marginparwidth=1.75cm]{geometry}
\usepackage{amsmath}
\usepackage{graphicx}
\usepackage[colorlinks=true, allcolors=blue]{hyperref}

\title{Отчёт по лабораторной работе № 22 по курсу "Алгоритмы и структуры данных"}
\author{}
\begin{document}
\maketitle
\raggedright
Студент группы: М80-108Б-22 Галкин Алексей Дмитриевич, № по списку 3 \\
Контакты e-mail: alexgalkin2004@mail.ru \\
Работа выполнена: «1» апреля 2023 г. \\
Преподаватель: асп. каф. 806 Сахарин Никита Александрович \\
Входной контроль знаний с оценкой: 5 \\
Отчет сдан «8» апреля 2023  г., итоговая оценка 5 (отлично) \\
Подпись преподавателя:  \underline{\hspace{3cm}} \\


\section{Тема}
Издательская система \TeX

\section{Цель работы}.
Изучение и освоение системы \LaTeX

\section{Задание}
Составить отчёт по лабораторной работе в \LaTeX. Написать формулу, таблицу и вставить картинку.

\section{Оборудование}
Процессор: 11th Gen Intel(R) Core (TM) i7-11370H 3.30GHz \\
ОП: 16ГБ \\
Монитор: 2880x1800 \\
\section{Програмное обеспечение}
Опереционная система семейства: VirtualBox 6.1.38 - Ubuntu 22.04.01 LTS \\
Интерпретатор команд: -- \\ 
Система программирования: --, версия --  \\
Редактор текстов: -- \\
Утилиты операционной системы: -- \\
Прикладные системы и программы:-- \\
Местанохождение и имена файлов программ и данных на домашнем компьютере: home/alexey \\

\section{Идея, метод, алгоритм }   
\begin{enumerate}
\item Посмотреть в тестовый файл
\item Написать отчёт
\item Вставить в отчёт таблицы, формулы, картинки, чтобы показать особенности LaTeX
\end{enumerate}

\section{Сценарий выполнения работы} 
\begin{itemize}
\item Изучение особенностей \LaTeX
\item Написать отчёт в \LaTeX
\item Перенести отчёт в репозиторий в github
\end{itemize} 

Пункты 1-7 отчета составляются сторого до начала лабораторной работы. Допущен к выполнению работы.
Подпись преподавателя: 

\section{Распечатка протокола}

\[S_n = \frac{X_1 + X_2 + \cdots + X_n}{n}
      = \frac{1}{n}\sum_{i}^{n} X_i\]
\vspace{0.5cm}
\[
  A_{2\times2} =
  \left[ {\begin{array}{cc}
    a_{11} & a_{12} \\
    a_{21} & a_{22} \\
  \end{array} } \right]
\]
\vspace{0.5cm}
\[
  B_{m\times n} =
  \left[ {\begin{array}{cccc}
    b_{11} & b_{12} & \cdots & b_{1n}\\
    b_{21} & b_{22} & \cdots & b_{2n}\\
    \vdots & \vdots & \ddots & \vdots\\
    b_{m1} & b_{m2} & \cdots & b_{mn}\\
  \end{array} } \right]
\]

\vspace{2cm}

\centering
\includegraphics[scale=0.25]{phoenix.jpg}

\raggedright


\section{Дневник отладки}

\begin{tabular}{|c|p{1cm}|p{1.5cm}|c|p{2.5cm}|p{2cm}|p{2.25cm}|}
    \hline
    № & Лаб. или дом. & Дата & Время & Событие & Действие по исправлению & Примечание\\
    \hline
    1 & дом. & 01.04.2023 & 9:00 & Создание файла & - & - \\ 
    \hline
    2 & дом. & 01.04.2023 & 10:00 & Добавление таблицы & - & - \\ 
    \hline
    3 & дом. & 01.04.2023 & 10:20 & Добавление формулы & - & - \\ 
    \hline
\end{tabular}


\section{Замечания автора по существу работы}
По ходу защиты лабораторной работы было дано задание из Codeforces: \\
Защита с Codeforces (div 4): \\
https://codeforces.com/contest/1807/submission/198120827 \\
https://codeforces.com/contest/1807/submission/198217471 \\

Дорешка: \\
https://codeforces.com/contest/1807/submission/201158351 \\
https://codeforces.com/contest/1807/submission/201162280 \\

\section{Выводы}
Была изучена издательская система \TeX, освоен способ создания таблицы, формулы, вставки картинки. Эта лабораторная работа продемонстрировала все преимущества данной системы и показала, что большинство курсовых работ могут быть написаны в \LaTeX

\end{document}
